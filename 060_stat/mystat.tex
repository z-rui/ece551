\input cwebmac


% Sometimes I just cite the text in README.
% I make it narrower so it is more obvious.
\def\cite{\par\smallskip \leftskip=.5in \rightskip=.5in}


\N{1}{1}Introduction.
This is the source for assignment 060.
Here's some instructions from README:

{\cite
In this assignment, you will be producing your own, simplified version
of the UNIX utility \.{stat}, which provides information about a file.
\dots
Your ultimate
goal is to write a program called \.{mystat} which replicates the
functionality of the \.{stat} command without any options (that is,
you aren't going to do \.{-t}, \.{-c}, etc).  If you run \.{stat something}
and \.{mystat something} they should produce exactly the same output.

}

\fi

\M{2}Below are some header files already included in \.{mystat.c}.
I may want to include more when needed.

\Y\B\4\X2:headers\X${}\E{}$\6
\8\#\&{include} \.{<sys/types.h>}\6
\8\#\&{include} \.{<sys/stat.h>}\6
\8\#\&{include} \.{<time.h>}\6
\8\#\&{include} \.{<stdio.h>}\6
\8\#\&{include} \.{<stdlib.h>}\6
\8\#\&{include} \.{<ctype.h>}\6
\8\#\&{include} \.{<pwd.h>}\6
\8\#\&{include} \.{<grp.h>}\6
\8\#\&{include} \.{<unistd.h>}\par
\A6.
\U4.\fi

\M{3}And a \PB{\\{time2str}} function is provided, too.

\Y\B\4\X3:definitions\X${}\E{}$\SHC{This function is for Step 4 }\6
\&{char} ${}{*}{}$\\{time2str}(\&{const} \&{time\_t} ${}{*}\\{when},\39{}$%
\&{long} \\{ns})\1\1\2\2\6
${}\{{}$\1\6
\&{char} ${}{*}\\{ans}\K\\{malloc}(\T{128}*{}$\&{sizeof} ${}({*}\\{ans}));{}$\6
\&{char} \\{temp1}[\T{64}];\6
\&{char} \\{temp2}[\T{32}];\6
\&{const} \&{struct} \&{tm} ${}{*}\|t\K\\{localtime}(\\{when});{}$\7
${}\\{strftime}(\\{temp1},\39\T{512},\39\.{"\%Y-\%m-\%d\ \%H:\%M:\%S"},\39%
\|t);{}$\6
${}\\{strftime}(\\{temp2},\39\T{32},\39\.{"\%z"},\39\|t);{}$\6
${}\\{snprintf}(\\{ans},\39\T{128},\39\.{"\%s.\%09ld\ \%s"},\39\\{temp1},\39%
\\{ns},\39\\{temp2});{}$\6
\&{return} \\{ans};\6
\4${}\}{}$\2\par
\As7, 15, 18\ETs25.
\U4.\fi

\M{4}The whole program will look like this:

\Y\B\X2:headers\X\6
\X14:declarations\X\6
\X3:definitions\X\6
\X5\*:main program\X\par
\fi

\N{1}{5\*}Step 1.  As in README:

{\cite
Your first goal is to make your mystat program accept exactly ONE
filename as a command line argument, and print out the first
THREE lines of the output that \.{stat} would print for that file.

}

In Step~1, I wrote a main function that only needs to process \PB{\\{argv}[%
\T{1}]}.

In Step~5, README says
{\cite
Up to this point, your program has taken ONE command line argument,
however, the real stat program takes an arbitrary number of arguments,
and stats each one.  For example, you can run \.{stat README . /dev/null}
and it will print the information for all three files one after
the other.

}

So finally the \PB{\\{main}} function here is able to handle multiple
arguments.

\Y\B\4\X5\*:main program\X${}\E{}$\6
\&{int} \\{main}(\&{int} \\{argc}${},\39{}$\&{char} ${}{*}{*}\\{argv}){}$\1\1\2%
\2\6
${}\{{}$\1\6
\&{int} \|i;\6
\&{int} \\{rc}${}\K\.{EXIT\_SUCCESS};{}$\7
\&{if} ${}(\\{argc}<\T{2}){}$\5
${}\{{}$\C{ \.{stat} is run without arguments }\1\6
${}\\{fprintf}(\\{stderr},\3{-1}\39\.{"stat:\ missing\ opera}\)\.{nd\\n"}\3{-1}%
\.{"Try\ 'stat\ --help'\ f}\)\.{or\ more\ information.}\)\.{\\n"});{}$\6
\&{return} \.{EXIT\_FAILURE};\6
\4${}\}{}$\2\6
\&{for} ${}(\|i\K\T{1};{}$ ${}\|i<\\{argc};{}$ ${}\|i\PP){}$\5
${}\{{}$\1\6
\&{if} ${}(\\{printstat}(\\{argv}[\|i])\I\T{0}){}$\5
${}\{{}$\1\6
${}\\{rc}\K\.{EXIT\_FAILURE}{}$;\C{ \.{stat} returns failure if any argument
caused an error }\6
\4${}\}{}$\2\6
\4${}\}{}$\2\6
\&{return} \\{rc};\6
\4${}\}{}$\2\par
\U4.\fi

\M{6}Before proceeding, I think there are some headers very useful.

\Y\B\4\X2:headers\X${}\mathrel+\E{}$\6
\8\#\&{include} \.{<assert.h>}\C{ \PB{\\{assert}} }\6
\8\#\&{include} \.{<string.h>}\C{ \PB{\\{strcpy}}, \PB{\\{strerror}}, etc. }\6
\8\#\&{include} \.{<errno.h>}\C{ \PB{\\{errno}} }\par
\fi

\M{7}\PB{\\{printstat}} is the \.{stat} program for a single file.
For example, passing \PB{\.{"README"}} to it might produce output like this:
{\smallskip\tt\settabs 4\columns
\+File: `README'\cr
\+Size: 2825&           Blocks: 8&      IO Block: 4096& regular file\cr
\+Device: fc00h/64512d& Inode: 801829&  Links: 1\cr
\smallskip}

\noindent{\bf Note}\quad
The README said I should use the fancy quotes ({\tt `'}).
However, this specification is later changed (see Piazza @366),
and I should use ordinary quotes ({\tt \char13\char13}) instead.

\Y\B\4\X3:definitions\X${}\mathrel+\E{}$\6
\&{int} \\{printstat}(\&{const} \&{char} ${}{*}\\{filename}){}$\1\1\2\2\6
${}\{{}$\1\6
\&{struct} \&{stat} \\{st};\7
\X8:read stats into \PB{\\{st}}\X\6
\X9\*:print formatted stats\X\6
\&{return} \T{0};\C{ no error happened }\6
\4${}\}{}$\2\par
\fi

\M{8}I can make use of the \PB{\\{lstat}} system call to read the stats
into a \PB{\&{struct} \&{stat}}.

Before printing anything, I need to check if the call succeeded.
(Things like nonexistent file or permission denied might happen.)

\Y\B\4\X8:read stats into \PB{\\{st}}\X${}\E{}$\6
\&{if} ${}(\\{lstat}(\\{filename},\39{\AND}\\{st})\I\T{0}){}$\5
${}\{{}$\C{ \PB{\\{lstat}} returns non-zero on error }\1\6
${}\\{fprintf}(\\{stderr},\39\.{"stat:\ cannot\ stat\ '}\)\.{\%s':\ \%s\\n"},%
\39\\{filename},\39\\{strerror}(\\{errno}));{}$\6
\&{return} \\{errno};\C{ don't call \PB{\\{exit}}; real \.{stat} program
proceeds        and deals with other files }\6
\4${}\}{}$\2\par
\U7.\fi

\M{9\*}On the first line I only need to print the filename.

But (in step 7) when the file is a symbolic link,
I should also output what it links to.
\Y\B\4\X9\*:print formatted stats\X${}\E{}$\6
\&{if} ${}(\.{S\_ISLNK}(\\{st}.\\{st\_mode})){}$\5
${}\{{}$\C{ file is a symbolic link }\1\6
\&{char} \\{buf}[\T{256}];\6
\&{ssize\_t} \\{len};\7
${}\\{len}\K\\{readlink}(\\{filename},\39\\{buf},\39\T{256});{}$\6
${}\\{assert}(\\{len}\G\T{0}){}$;\C{ negative length does not make sense }\6
\&{if} ${}(\\{len}>\T{255}){}$\5
${}\{{}$\1\6
${}\\{buf}[\T{255}]\K\.{'\\0'};{}$\6
\4${}\}{}$\2\6
\&{else}\5
${}\{{}$\1\6
${}\\{buf}[\\{len}]\K\.{'\\0'};{}$\6
\4${}\}{}$\2\6
${}\\{printf}(\.{"\ \ File:\ '\%s'\ ->\ '\%s}\)\.{'\\n"},\39\\{filename},\39%
\\{buf});{}$\6
\4${}\}{}$\2\6
\&{else}\5
${}\{{}$\1\6
${}\\{printf}(\.{"\ \ File:\ '\%s'\\n"},\39\\{filename});{}$\6
\4${}\}{}$\2\par
\As10, 11\*, 12\ETs23.
\U7.\fi

\M{10}On the second line I need to print `Size', `Blocks', and `IO Block'.

\Y\B\4\X9\*:print formatted stats\X${}\mathrel+\E{}$\6
${}\{{}$\1\6
\&{const} \&{char} ${}{*}\\{filetype}\K\NULL;{}$\7
\&{switch} ${}(\\{st}.\\{st\_mode}\AND\.{S\_IFMT}){}$\5
${}\{{}$\1\6
\4\&{case} \.{S\_IFBLK}:\5
${}\\{filetype}\K\.{"block\ special\ file"}{}$;\5
\&{break};\6
\4\&{case} \.{S\_IFCHR}:\5
${}\\{filetype}\K\.{"character\ special\ f}\)\.{ile"}{}$;\5
\&{break};\6
\4\&{case} \.{S\_IFDIR}:\5
${}\\{filetype}\K\.{"directory"}{}$;\5
\&{break};\6
\4\&{case} \.{S\_IFIFO}:\5
${}\\{filetype}\K\.{"fifo"}{}$;\5
\&{break};\6
\4\&{case} \.{S\_IFLNK}:\5
${}\\{filetype}\K\.{"symbolic\ link"}{}$;\5
\&{break};\6
\4\&{case} \.{S\_IFREG}:\5
${}\\{filetype}\K\.{"regular\ file"}{}$;\5
\&{break};\6
\4\&{case} \.{S\_IFSOCK}:\5
${}\\{filetype}\K\.{"socket"}{}$;\5
\&{break};\6
\4${}\}{}$\2\6
${}\\{assert}(\\{filetype}\I\NULL){}$;\C{ none of the possibilities described
in README! I'm confused\dots }\6
${}\\{printf}(\.{"\ \ Size:\ \%-10lu\\tBlo}\)\.{cks:\ \%-10lu\ IO\ Block}\)\.{:%
\ \%-6lu\ \%s\\n"},\3{-1}\39\\{st}.\\{st\_size},\39\\{st}.\\{st\_blocks},\39%
\\{st}.\\{st\_blksize},\39\\{filetype});{}$\6
\4${}\}{}$\2\par
\fi

\M{11\*}On the third line, I need to print `Device', `Inode', and `Links'.

Additionally (in step 6), another field called ``Device type''
will be appended to the third line if the file happens to be
a device.

\Y\B\4\X9\*:print formatted stats\X${}\mathrel+\E{}$\6
$\\{printf}(\.{"Device:\ \%lxh/\%lud\\t}\)\.{Inode:\ \%-10lu"},\39\\{st}.\\{st%
\_dev},\39\\{st}.\\{st\_dev},\39\\{st}.\\{st\_ino});{}$\6
\&{if} ${}(\.{S\_ISCHR}(\\{st}.\\{st\_mode})\V\.{S\_ISBLK}(\\{st}.\\{st%
\_mode})){}$\5
${}\{{}$\C{ file is a device }\1\6
${}\\{printf}(\.{"\ \ Links:\ \%-5lu\ Devi}\)\.{ce\ type:\ \%d,\%d\\n"},\3{-1}%
\39\\{st}.\\{st\_nlink},\39\\{major}(\\{st}.\\{st\_rdev}),\39\\{minor}(\\{st}.%
\\{st\_rdev}));{}$\6
\4${}\}{}$\2\6
\&{else}\5
${}\{{}$\1\6
${}\\{printf}(\.{"\ \ Links:\ \%lu\\n"},\39\\{st}.\\{st\_nlink});{}$\6
\4${}\}{}$\2\par
\fi

\N{1}{12}Step 2.  Now I need to deal with the fourth line
in the formatted output.
To leave room for future extension,
I will not hard-code what to print here.

\Y\B\4\X9\*:print formatted stats\X${}\mathrel+\E{}$\6
\X13:print the fields in the fourth formatted line\X\6
\\{printf}(\.{"\\n"});\C{ terminate the fourth line }\par
\fi

\M{13}At this point, only the first field (``Access'') needs to be output
on the fourth line.
It consists of two parts.  The first is an octal code of the permissions.
I need to mask \PB{\\{st\_mode}} with \PB{$\CM\.{S\_IFMT}$} to get that code.
The second part is a human-readable description
(I will call it ``mode string'' in later descriptions),
which can be generated by \PB{\\{get\_modestr}}.

\Y\B\4\D$\.{MODESTR\_LEN}$ \5
\T{10}\C{ length of the mode string, not including \PB{\.{'\\0'}} }\par
\Y\B\4\X13:print the fields in the fourth formatted line\X${}\E{}$\6
${}\{{}$\1\6
\&{char} ${}\\{modestr}[\.{MODESTR\_LEN}+\T{1}]{}$;\C{ ${}+1$ for \PB{\.{'%
\\0'}} }\7
${}\\{printf}(\.{"Access:\ (\%04o/\%s)"},\39\\{st}.\\{st\_mode}\AND\CM\.{S%
\_IFMT},\39\\{get\_modestr}(\\{modestr},\39\\{st}.\\{st\_mode}));{}$\6
\4${}\}{}$\2\par
\As21\ET22.
\U12.\fi

\M{14}\PB{\\{get\_modestr}} generates the mode string.

\Y\B\4\X14:declarations\X${}\E{}$\6
\&{char} ${}{*}{}$\\{get\_modestr}(\&{char} ${}{*}\\{buf},\39{}$\&{int} %
\\{mode});\par
\As17\ET24.
\U4.\fi

\M{15}\B\X3:definitions\X${}\mathrel+\E{}$\6
\&{char} ${}{*}{}$\\{get\_modestr}(\&{char} ${}{*}\\{buf},\39{}$\&{int} %
\\{mode})\1\1\2\2\6
${}\{{}$\1\6
\X16:write file type into \PB{\\{buf}[\T{0}]}\X\6
\X19:write permissions into \PB{\\{buf}[\T{1}]}\dots\PB{$\\{buf}[\.{MODESTR%
\_LEN}-\T{1}]$}\X\6
${}\\{buf}[\.{MODESTR\_LEN}]\K\.{'\\0'}{}$;\C{ terminate the string }\6
\&{return} \\{buf};\6
\4${}\}{}$\2\par
\fi

\M{16}The first character in the mode string
will need a big switch
similar to the one in Step 1.
But some characters are different, so I cannot reuse the code.

\Y\B\4\X16:write file type into \PB{\\{buf}[\T{0}]}\X${}\E{}$\6
${}\{{}$\1\6
\&{char} \|c${}\K\T{0};{}$\7
\&{switch} ${}(\\{mode}\AND\.{S\_IFMT}){}$\5
${}\{{}$\1\6
\4\&{case} \.{S\_IFBLK}:\5
${}\|c\K\.{'b'}{}$;\5
\&{break};\6
\4\&{case} \.{S\_IFCHR}:\5
${}\|c\K\.{'c'}{}$;\5
\&{break};\6
\4\&{case} \.{S\_IFDIR}:\5
${}\|c\K\.{'d'}{}$;\5
\&{break};\6
\4\&{case} \.{S\_IFIFO}:\5
${}\|c\K\.{'p'}{}$;\5
\&{break};\6
\4\&{case} \.{S\_IFLNK}:\5
${}\|c\K\.{'l'}{}$;\5
\&{break};\6
\4\&{case} \.{S\_IFREG}:\5
${}\|c\K\.{'-'}{}$;\5
\&{break};\6
\4\&{case} \.{S\_IFSOCK}:\5
${}\|c\K\.{'s'}{}$;\5
\&{break};\6
\4${}\}{}$\2\6
${}\\{assert}(\|c\I\T{0}){}$;\C{ Again, I'm confused when run out of
possibilities. }\6
${}\\{buf}[\T{0}]\K\|c{}$;\C{ write it into \PB{\\{buf}[\T{0}]} }\6
\4${}\}{}$\2\par
\U15.\fi

\M{17}\PB{\\{get\_permissions}} unifies the formatting
of user, group and other permissions.

\Y\B\4\X14:declarations\X${}\mathrel+\E{}$\6
\&{void} \\{get\_permissions}(\&{char} ${}{*}\\{buf},\39{}$\&{int} \\{mode}${},%
\39{}$\&{int} \\{rmask}${},\39{}$\&{int} \\{wmask}${},\39{}$\&{int} \\{xmask});%
\par
\fi

\M{18}\B\X3:definitions\X${}\mathrel+\E{}$\6
\&{void} \\{get\_permissions}(\&{char} ${}{*}\\{buf},\39{}$\&{int} \\{mode}${},%
\39{}$\&{int} \\{rmask}${},\39{}$\&{int} \\{wmask}${},\39{}$\&{int} \\{xmask})%
\1\1\2\2\6
${}\{{}$\1\6
${}\\{buf}[\T{0}]\K(\\{mode}\AND\\{rmask})\?\.{'r'}:\.{'-'};{}$\6
${}\\{buf}[\T{1}]\K(\\{mode}\AND\\{wmask})\?\.{'w'}:\.{'-'};{}$\6
${}\\{buf}[\T{2}]\K(\\{mode}\AND\\{xmask})\?\.{'x'}:\.{'-'};{}$\6
\4${}\}{}$\2\par
\fi

\M{19}Now I can call \PB{\\{get\_permissions}} to fill in the corresponding
fields
in the mode string.

\Y\B\4\X19:write permissions into \PB{\\{buf}[\T{1}]}\dots\PB{$\\{buf}[%
\.{MODESTR\_LEN}-\T{1}]$}\X${}\E{}$\6
$\\{get\_permissions}(\\{buf}+\T{1},\39\\{mode},\39\.{S\_IRUSR},\39\.{S%
\_IWUSR},\39\.{S\_IXUSR});{}$\6
${}\\{get\_permissions}(\\{buf}+\T{4},\39\\{mode},\39\.{S\_IRGRP},\39\.{S%
\_IWGRP},\39\.{S\_IXGRP});{}$\6
${}\\{get\_permissions}(\\{buf}+\T{7},\39\\{mode},\39\.{S\_IROTH},\39\.{S%
\_IWOTH},\39\.{S\_IXOTH}){}$;\par
\U15.\fi

\N{1}{20}Step 3.  Now deal with the other fields in the fourth line.
They are ``Uid'' and ``Gid''.

\fi

\M{21}First deal with ``Uid''.
I can use the \PB{\\{getpwuid}} library function
to look up the user's name from it's user~ID.

Be careful about the leading spaces in the output.
They are here to separate the current field from
the previous one.

\Y\B\4\X13:print the fields in the fourth formatted line\X${}\mathrel+\E{}$\6
${}\{{}$\1\6
\&{struct} \&{passwd} ${}{*}\\{pw};{}$\7
${}\\{pw}\K\\{getpwuid}(\\{st}.\\{st\_uid});{}$\6
${}\\{printf}(\.{"\ \ Uid:\ (\%5d/\%8s)"},\39\\{st}.\\{st\_uid},\39\\{pw}\MG%
\\{pw\_name});{}$\6
\4${}\}{}$\2\par
\fi

\M{22}Then ``Gid''.  This time use \PB{\\{getgrgid}}.

\Y\B\4\X13:print the fields in the fourth formatted line\X${}\mathrel+\E{}$\6
${}\{{}$\1\6
\&{struct} \&{group} ${}{*}\\{gr};{}$\7
${}\\{gr}\K\\{getgrgid}(\\{st}.\\{st\_gid});{}$\6
${}\\{printf}(\.{"\ \ \ Gid:\ (\%5d/\%8s)"},\39\\{st}.\\{st\_gid},\39\\{gr}\MG%
\\{gr\_name});{}$\6
\4${}\}{}$\2\par
\fi

\N{1}{23}Step 4.
In this step I will add four more lines to the output.
They are the last ones, and are related to time information.

Again, these lines have some similarities,
so I will put the actual printing code
into a separate function \PB{\\{print\_timeinfo}}.

\Y\B\4\X9\*:print formatted stats\X${}\mathrel+\E{}$\6
$\\{print\_timeinfo}(\.{"Access"},\39{\AND}\\{st}.\\{st\_atim});{}$\6
${}\\{print\_timeinfo}(\.{"Modify"},\39{\AND}\\{st}.\\{st\_mtim});{}$\6
${}\\{print\_timeinfo}(\.{"Change"},\39{\AND}\\{st}.\\{st\_ctim});{}$\6
\\{printf}(\.{"\ Birth:\ -\\n"});\C{ ``Birth'' line is always this }\par
\fi

\M{24}\B\X14:declarations\X${}\mathrel+\E{}$\6
\&{void} \\{print\_timeinfo}(\&{const} \&{char} ${}{*}\\{what},\39{}$\&{const} %
\&{struct} \&{timespec} ${}{*}\\{when}){}$;\par
\fi

\M{25}\PB{\\{print\_timeinfo}} uses the provided \PB{\\{time2str}} function
to format the time value into a string.
\Y\B\4\X3:definitions\X${}\mathrel+\E{}$\6
\&{void} \\{print\_timeinfo}(\&{const} \&{char} ${}{*}\\{what},\39{}$\&{const} %
\&{struct} \&{timespec} ${}{*}\\{when}){}$\1\1\2\2\6
${}\{{}$\1\6
\&{char} ${}{*}\\{timestr};{}$\7
${}\\{timestr}\K\\{time2str}({\AND}\\{when}\MG\\{tv\_sec},\39\\{when}\MG\\{tv%
\_nsec});{}$\6
${}\\{printf}(\.{"\%s:\ \%s\\n"},\39\\{what},\39\\{timestr});{}$\6
\\{free}(\\{timestr});\6
\4${}\}{}$\2\par
\fi

\N{1}{26}Step 5, 6, 7.
These steps require modifications to previous steps.

The modifications are already in place.
So no code will be shown here.

An asterisk next to a section number indicates modification.


\fi

\N{1}{27\*}Index.  Here is the index of all C identifiers, in alphabetical
order.
The listed numbers are section numbers.
The underlined ones are where the corresponding identifiers are defined.


% vim: ts=2:sw=2:et:si
\fi

\ch 5\*, 9\*, 11\*, 27\*.
\inx
\fin
\con
